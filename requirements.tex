\chapter{Requirements}
\section{Requirements}
\label{requirements}
\subsection{Functional Requirements}
\subsubsection{Critical}
  \paragraph{Efficacy} Precise \& Accurate Identification

  A system based mainly on object-detection capabilities is only as useful as the accuracy of the resulting data. Most of the popular object-detection frameworks that currently exist display results based on a certain confidence threshold generated by the model itself. Therefore, it is imperative that this threshold be set to a very high value, in order to limit the number of false positive results. On the other hand, false negatives are similarly problematic and contingent on the training data used to train the model.

  \paragraph{Speed} Fast Inference Speed

The term \emph{inference} refers to the speed at which a computing device can process a set of data and output a result using a mathematical model. In this project, the ``sets of data'' that will be processed are digital images representing each frame of the video. The model that we will employ is likely to be one of the many industry standard deep learning models that are commonly used for computer vision. Inference speed in such a scenario is more accurately represented as the number of video frames that can be processed per second. Due to the state of low-power computing and wireless communication at this time, it is likely not feasible to display this video feed in real time to users, but every extra frame that can be captured and processed each second makes the resulting data more accurate and provides more data points to first responders.
  \paragraph{Notification} Real-Time Notification of Detection

As mentioned above, the power usage and computing overhead necessary to transmit an annotated video stream to users is likely to high to be technically feasible in this application. However, the data that will be generated by the imaging array on each UAV is incredibly useful to disaster response teams. As a result, it was deemed a critical requirement that the system must provide instantaneous notification to users when a human (or other object class depending on circumstances) is detected by the system. This notification allows the disaster response teams to modify their flight path or, where necessary, take manual control of the flight in order to procure more data that may save additional lives.
  \paragraph{Power} Lowest Possible Power Footprint

The maximum flight time of the UAV is completely contingent on the battery capacity available to its DC motors while in flight. Therefore, it is imperative that any auxiliary functions require as little power as possible. Preliminary calculations showed us that the maximum power draw of most single board computers that exist today are significantly lower than that of drone flight, but every optimization is useful, and incredibly important, when such a system is deployed into a disaster scenario.
\subsubsection{Recommended}
  \paragraph{Metrics} Count the number of victims that have been identified along a flight path
  \paragraph{UAV Cooperation}

Allow for multiple UAV's to be used simultaneously during a disaster response scenario
\subsubsection{Suggested}

