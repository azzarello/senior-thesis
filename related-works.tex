\chapter{Related Works}

\section{Related Work}
    % a summary of similar works
    % try to identify their shortcomings

In creating our solution, we decided to draw inspiration from other studies already produced in the field of object detection and computer vision. By doing extensive research on other projects, we were able to narrow down on the best dataset to use as well. 
In the first project we looked at, we discovered the company Nanonets that specializes in making convolutional neural networks for aerial drones. As it turns out, drones are used for a variety of industry applications such as surveillance over solar panel farms, monitoring progress in construction projects, and various other cases. We were able to gather some information about the algorithms used but could not apply it heavily to our project as they didn’t have any written documentation on human detection. 
A larger scale project we found was one from a Microsoft team, which created a drone system of manned aerial vehicles (MAVs) to persistently track one individual. It could be for the purposes of tracking or other security-based applications. Like our project, the Microsoft team implemented an approach that avoided performing detection on every frame, and instead, focused on creating a performance-effective pattern detection hypothesis on frames which, if deemed to be important, would then have detection performed on the relevant frames. We followed this approach since applying detection per frame for our system would be very power intensive, and disallow for longer flight times or the redirection of power to our embedded systems. 
An important related work in our process to finding a reliable imaging dataset was a paper by TODO which outlined different available training datasets for projects that could not afford the time or means to create their own. In the article are many notable examples like Haar Cascade and MobileNet. It dives into the differences and similarities between the two, the situations where one is better than the other, and implementation required. 
% TODO write something technical here to finish up this paragraph about our dataset>.
% We also searched and compared various drone hardware designs to find the best approach to how to create our drone. We were able to recognize that creating a low cost drone could be achieved after looking into the work by ___ from Indiana University. The team was able to create a cheaper \$300 drone with a 1GHz CPU and two front-facing cameras, and still support an average computer vision model. 
% 
